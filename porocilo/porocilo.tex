\documentclass[11pt,a4paper]{article}

\usepackage[slovene]{babel}
\usepackage[utf8x]{inputenc}
\usepackage{graphicx}

\pagestyle{plain}

\begin{document}
\title{Poročilo pri predmetu \\
Analiza podatkov s programom R}
\author{Urška Bele}
\maketitle

\section{Izbira teme}
Za temo mojega projekta sem izbrala Olimpijske igre moderne dobe.
V projektu bom za posamezne olimpijske igre (poletne in zimske) v dveh tabelah navedla različne podatke.
Prva tabela (v HTML obliki) vsebuje podatke: mesto (imenska spremenljivka); država (imeska); kontinent (imeska); 
zaporedna številka; vrsta (imeska); leto (številska). Tabela: http://en.wikipedia.org/wiki/List_of_Olympic_Games_host_cities.
Druga tabela (v CSV obliki) vsebuje podatke: vrsta OI (imenska spremenljivka); število držav udeleženk (številska); 
število športnikov (številska); razsežnost glede na število držav udeleženk (urejenostna); število športov (številska); 
število dogodkov (številska); država z največ medaljami (številska).
Cilji: V projektu bom na podlagi zgornjih podatkov lahko določila državo, ki je OI priredila največkrat. 
Izračunala bom maksimalno, minimalno in povprečno število držav udeleženk, disciplin in prejetih medalj. 
Olimpijske igre bom razvrstila v kategorije razsežnosti glede na število udeleženk. Dobljene rezultate bom prikazala tudi 
na zemljevidu.
Podatki: Nekatere podatke sem dobila že zbrane v tabeli na spletni strani 
http://en.wikipedia.org/wiki/List_of_Olympic_Games_host_cities. Podatke za drugo tabelo sem pridobila s spletnih strani
 Wikipedije, kjer obstaja članek za vsake olimpijske igre posebej 
 (npr.http://en.wikipedia.org/wiki/2012_Summer_Olympics). 
 Podatke sem zbrala v tabelo in jo shranila kot CSV (naložena na repozitoriju), da jo bom lahko uvozila v R 
 in obdelala podatke.

\section{Obdelava, uvoz in čiščenje podatkov}

\section{Analiza in vizualizacija podatkov}

\includegraphics{../slike/povprecna_druzina.pdf}

\section{Napredna analiza podatkov}

\includegraphics{../slike/naselja.pdf}

\end{document}
