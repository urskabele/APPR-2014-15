\documentclass[11pt,a4paper]{article}

\usepackage[slovene]{babel}
\usepackage[utf8x]{inputenc}
\usepackage{graphicx}
\usepackage{hyperref}
\usepackage{pdfpages}

\pagestyle{plain}

\begin{document}
\title{Poročilo pri predmetu \\
Analiza podatkov s programom R}
\author{Urška Bele}
\maketitle

\section{Izbira teme}

Za temo mojega projekta sem izbrala Olimpijske igre moderne dobe.
V projektu bom za posamezne olimpijske igre (poletne in zimske) v treh tabelah navedla različne podatke. \newline
\textbf{Prva tabela} (v CSV obliki) vsebuje podatke:
\begin{enumerate}
\item{\verb|Leto| (številska spremenljivka),}
\item{\verb|Vrsta| - poletne ali zimske (imenska spremenljivka),}
\item{ \verb|Število udeleženk| (številska) - število držav, ki so sodelovale na OI,}
\item{\verb|Število športnikov| (številska) - skupno število športnikov,}
\item{\verb|Razsežnost| glede na število držav udeleženk (urejenostna),}
\item{\verb|Število športov| (številska),}
\item{\verb|Število dogodkov| (številska) - športi se delijo na discipline, vsaka disciplina pa je predstavljena na več dogodkih, npr. tenis moških posameznikov, ženskih posameznic, moških dvojic,...,}
\item{\verb|Država z največ medaljami| (imenska).}
\end{enumerate}
Podatke za prvo tabelo sem pridobila s spletnih strani
 Wikipedije, kjer obstaja članek za vsake olimpijske igre posebej, npr.
\url{http://en.wikipedia.org/wiki/2012_Summer_Olympics}. 
 Podatke sem zbrala v tabelo in jo shranila kot CSV (naložena na repozitoriju), da jo bom lahko uvozila v R 
 in obdelala podatke.
\newline
\textbf{Druga tabela}, prav tako v CSV obliki, vsebuje podatke o številu športnikov iz posameznih držav, ki so se udeležili 
olimpijskih iger leta 2012 v Londonu. Podatke sem dobila na spletni strani \url{http://en.wikipedia.org/wiki/2012_Summer_Olympics} in jih shranila v tabelo v CSV obliki, ki jo bom kasneje uvozila v R.
\newline
\textbf{Tretja tabela} (v HTML obliki) vsebuje podatke: 
\begin{enumerate}
\item{\verb|Mesto| (imenska spremenljivka),}
\item{\verb|Država| (imeska),}
\item{\verb|Kontinent| (imeska),}
\item{\verb|Vrsta| (imeska),}
\item{\verb|Leto| (številska),}
\item{\verb|Datum začetka|,}
\item{\verb|Datum zaključka|.}

\end{enumerate}
 Tabela:
\url{http://en.wikipedia.org/wiki/List_of_Olympic_Games_host_cities}.
\newline

\textbf{Cilji:} V projektu bom na podlagi zgornjih podatkov lahko opazovala trend udeleževanja OI. 
Izračunala bom maksimalno, minimalno in povprečno število športnikov, ki so jih države poslale na OI, disciplin in prejetih medalj. 
Olimpijske igre bom razvrstila v kategorije razsežnosti glede na število udeleženk. Dobljene rezultate bom prikazala tudi 
na zemljevidu. Poskusila bom napovedati podatke za prihodnost.

\newpage
\section{Obdelava, uvoz in čiščenje podatkov}
V drugi fazi sem najprej uvozila podatke, in sicer tri tabele. Prvo tabelo z imenom \verb|OI| sem uvozila iz CSV formata. Najprej sem v Excelu napisala tabelo in jo shranila kot CSV. Funkcija, ki to tabelo uvozi, se nahaja v mapi \verb|uvoz|, v datoteki \verb|uvoz.r|, imenuje pa se \verb|uvozi1|. Znak \verb|-| se interpretira kot \verb|NA|, prva vrstica pa je glava tabele. Podatki v tabeli so števila in nizi. Tej tabeli nisem rabila odstranjevati  ali spreminjati podatkov, saj je bila že urejena.
\newline
Na enak način sem uvozila tudi tabelo z imenom \verb|sportniki|.
\newline
Uvozila sem tudi HTML tabelo z imenom \verb|mesta|, in sicer najprej v datoteki \verb|xml.r| v mapi \verb|lib|. Na spletni strani je bilo več tabel, jaz sem uvozila prvo, iz nje naredila matriko in nato tabelo. Morala sem pobrisati nekatere stolpce in urediti podatke. Na koncu sem tudi to tabelo uvozila v datoteki \verb|uvoz.r| v mapi \verb|uvoz|.
\newline

Narisala sem dva tudi grafa. V mapi \verb|slike| sem ustvarila novo R-skripto. Najprej sem napisala v prvi in zadnji vrstici ukaze, ki mi uvozijo grafe v pdf obliko. Za oba grafa sem podatke vzela iz tabele \verb|OI|, saj so v drugi pretežno imenske spremenljivke.
Oba grafa sta vrste \verb|plot|. Prikazujeta število držav udeleženk na Olimpijskih igrah, prvi na poletnih in drugi na zimskih OI. Da sem ju narisala, sem iz stolpca \verb|Stevilo.udelezenk| ločeno pobrala podatke za zimske in poletne OI (to sem dobila iz stolpca \verb|Vrsta|). Grafa sem ločila, ker se mi primerjanje števila udeleženk na zimskih in poletnih OI ni zdelo smiselno.
\newline
Grafa sta prikazana spodaj, opazimo lahko trend naraščanja števila držav udeleženk tako na poletnih, kot tudi na zimskih olimpijskih igrah. V času hladne vojne je zaradi bojkotov število držav padlo v letih 1980 in 1984. Število držav je najhitreje naraščalo v drugi polovici 20. stoletja, najverjetneje predvsem zaradi nastajanja novih držav in razvoja tehnologije, ki je omogočila potovanja na OI tudi iz bolj oddaljenih držav. Od leta 2000 lahko opazimo, da število držav udeleženk narašča veliko počasneje. \newline
Število držav narašča tudi na zimskih OI, bolj sunkovita rast se je začela v 80. letih prejšnjega stoletja in se še ni umirila.
 

\includepdf[pages={1-2}]{../slike/grafi.pdf}

\section{Analiza in vizualizacija podatkov}
V tretji fazi sem na podlagi mojih podatkov narisala zemljevid. Odločila sem se za prikaz števila športnikov iz posameznih držav, ki so se udeležili OI 2012 v Londonu.
\newline
Program sem napisala v skripto \verb|vizualizacija.r| v mapi \verb|vizualizacija| in ga vključila v \verb|projekt.r|, da se pokliče z ukazom \verb|Source|.
\newline
Najprej sem s spletne strani \url{http://www.naturalearthdata.com/features/} v R uvozila zemljevid sveta z mejami držav. To sem naredila s pomočjo že napisane funkcije \verb|uvozi.zemljevid|, poimenovala sem ga \verb|svet|. Nato sem morala uskladiti 
podatke iz tabele \verb|sportniki| z zemljevidom, iz tabele \verb|sportniki| sem torej pobrala ven tiste države, ki so tudi na zemljevidu. Za nekatere države na zemljevidu podatkov v moji tabeli nisem imela. Naletela sem še na eno težavo, in sicer imena nekaterih držav se v moji tabeli s podatki in v zemljevidu niso ujemala (npr. Great Britain in United Kingdom). To sem rešila tako, da sem pri uvozu tabele \verb|sportniki| v skripti \verb|uvoz.r| preimenovala te države.
\newline
Nato sem s pomočjo izračunanega minimuma in maksimuma mojih podatkov ustrezno pobarvala zemljevid v odtenkih rdeče barve - temnejši odtenki predstavljajo države, ki so na OI poslale več športnikov, svetlejši pa tiste z manj športniki.
Dodala sem še naslov in legendo ter na zemljevidu s pomočjo funkcije \verb|coordinates| poimenovala večje države, označila nekaj večjih mest in jih poimenovala.
\newline
Zemljevid sem vključila v poročilo na naslednji strani. Po pričakovanjih so na OI največ športnikov poslale države z največ prebivalci (ZDA, Kitajska, Rusija). Veliko vlogo pa ima gospodarska razvitost: Indija, država z milijardo prebivalcev, ima dokaj nizko številko, prav tako večina afriških držav, medtem ko so številke evropskih držav precej visoke. Torej manj razvite države tudi manj sodelujejo na OI.

%\includegraphics[width=\textwidth]{../slike/zemljevid.pdf}
\makebox[\textwidth][c]{
\includegraphics[width=1.5\textwidth]{../slike/zemljevid.pdf}
}

\newpage
\section{Napredna analiza podatkov}
V četrti fazi projekta sem se ukvarjala z napredno analizo podatkov. V prvem delu sem zemljevid iz tretje faze naredila bolj reprezentativen tako, da sem upoštevala število prebivalcev v vsaki državi. V drugem delu sem opazovala trend števila dogodkov na OI in na podlagi različnih modelov poskusila napovedati, kaj se bo z njim dogajalo v prihodnjih letih. V tretjem delu pa sem poskusila države razvrstiti v smiselne skupine. Program sem napisala v skripto \verb|analiza.r|, ki se nahaja v mapi \verb|analiza|.
\newline
\begin{enumerate}
\item{\textbf{ZEMLJEVID}
\newline
V četrti fazi sem poskusila izboljšati zemljevid iz tretje faze. Na tem zemljevidu sem prikazala število športnikov na milijon prebivalcev posamezne države. Podatke o populaciji sem dobila iz že uvoženega zemljevida \verb|svet| in jih shranila v vektor \verb|pop|. Število športnikov sem pomnožila z milijon. S funkcijo \verb|match| sem uredila podatke, končni uporabljeni podatki so v vektorju \verb|podatki.za.norm.zemlj| in nato izračunala max, min, povprečje ter pobarvala zemljevid v odtenkih modre. Ta je shranjen v mapi \verb|slike| pod imenom \verb|normiranzemljevid.pdf|. Dodala sem legendo in imena večjih držav. Zemljevid sem vključila v poročilo spodaj.
\newline
Vidimo lahko, da se ta zemljevid precej razlikuje od prejšnjega. Temnejši odtenki predstavljajo države z večjim, svetlejši pa z manjšim številom športnikov na milijon prebivalcev. Največ (kar 88) šprtnikov na milijon prebivalcev je na OI sodelovalo iz Islandije, število je visoko tudi v Novi Zelandiji. Slovenija je poslala 32 športnikov na milijon prebivalcev. Visoko sta tudi Kanada in Avstralija in nekatere evropske države. Zaradi velikega števila prebivalcev so Kitajska, Rusija in ZDA svetleje obarvane, prav tako večina afriških držav.
}
\makebox[\textwidth][c]{
\includegraphics[width=1.5\textwidth]{../slike/normiranzemljevid.pdf}
}
\newpage
\item{\textbf{MODELI IN NAPOVED}
\newline
V drugem delu sem opazovala, kako hitro se spreminja število dogodkov na OI, saj je to zagotovo povezano s spreminjanjem števila udeleženih držav. Pričakujem, da bo naraščanje števila dogodkov podobno naraščanju števila držav udeleženk.
\newline
Najprej sem iz tabele \verb|OI| pobrala leta poletnih OI in jih shranila v \verb|letop|, število dogodkov pa v \verb|dogp|. Podobno za zimske OI, \verb|letoz| in \verb|dogz|. Nato sem za te podatke narisala grafa, kjer sem dejansko število dogodkov označila s pikicami. Potem sem za vsak graf narisala tri krivulje, ki se prilegajo podatkom. Linearna modela \verb|linp| in \verb|linz| sem na grafih označila z zeleno, kvadratna modela \verb|kvp| in \verb|kvz| pa z vijolično. S pomočjo ukazov \verb|linp$coefficients| in \verb|linz$coefficients| sem dobila naslednji enačbi premic:
$$y=1.980301x-3696.671186$$ za poletne olimpijske igre, kjer je $y$ število dogodkov in $x$ leto, ter $$y=0.906808x-1744.386889$$ za zimske olimpijske igre, kjer je $y$ število dogodkov in $x$ leto. Na podoben način, v \verb|kvp$coefficients| in \verb|kvp$coefficients| sem dobila tudi enačbi parabol: $$y=0.01262904x^2-47.41720x+44589.94$$ za poletne olimpijske igre, kjer je $y$ število dogodkov in $x$ leto, ter $$y=0.01146727x^2-44.29743x+42795.59$$ za zimske olimpijske igre, kjer je $y$ število dogodkov in $x$ leto.
Z oranžno sem narisala še modela \verb|loep| in \verb|loez|, za katera sem uporabila funkcijo \verb|loess| z lokalnim prileganjem. Osi, ki predstavljata leto OI, sem podaljšala do leta 2050, do tam torej segajo moje napovedi. Izračunala sem tudi ostanka \verb|ostp| in \verb|ostz|, ki sta povedala, da se podatkom najbolj prilegata modela \verb|loess|, najmanj pa linearna modela.
\newline
Iz dobljenih napovedi lahko vidimo, da naj bi se število dogodkov na OI še naprej povečevalo. Opazila sem tudi zelo podobno obliko grafov števila dogodkov in števila udeleženk iz 2. faze. Zato  lahko v prihodnosti pričakujemo tudi povečevanje števila udeleženk in seveda športnikov.
}
\includepdf[pages={1-2}]{../slike/naprednigrafi.pdf}
\newpage
\item{\textbf{RAZVRSTITEV V SKUPINE}
\newline
Na zemljevidih sem opazila, da razvite države načeloma na OI pošljejo več športnikov kot manj razvite. Zato sem v tej fazi narisala dendrogram in s tem preverila, kako se v skupine razvrstijo države. Najprej sem skonstruirala tabelo \verb|tab.za.skupine|, ki vsebuje imena držav in njihovo število športnikov na milijon prebivalcev. Nato sem s funkcijo \verb|scale| normalizirala podatke in narisala dendrogram. Države sem s funkcijo \verb|rect.hclust| razdelila v 6 skupin in dodala legendo z barvami. Skupina 1 predstavlja države z najmanj, skupina 6 pa države z največ športniki na milijon prebivalcev.
\newline
Ugotovila sem, da prvo skupino res sestavljajo manj razvite države, to so predvsem afriške in južnoameriške države ter države zahodne Azije (Afganistan, Pakistan, Indija,...). V tej skupini je seveda zaradi zelo velikega števila prebivalcev tudi Kitajska, ki je precej bolj razvita od ostalih držav v tej skupini. Drugo skupino res sestavljajo malo bolj razvite države, in sicer predvsem vzhodnoevropske in južnoameriške države. Glavne izjeme v tej skupini so ZDA, Francija, Nemčija in Japonska. V tretji skpupini so zelo različne države, od Armenije in Mongolije do Kanade in Velike Britanije, vseeno pa je večina teh držav nekoliko bolj razvitih od prejšnjih skupin. V četrto skupino, v katero spada tudi Slovenija, so uvrščene predvsem severnoevropske države ter Švica in Avstralija, pa tudi Jamajka in Trinidad in Tobago, ki na poletne OI pošiljata zelo veliko atletov, prebivalstvo pa ni zelo številčno. Peto in šesto skupino naj bi sestavljale najbolj razvite države, a tu je po mojem mnenju prišlo do največjih odstopanj. Črna Gora, Bahami in Islandija so namreč države z malo prebivalci (vse pod milijon), Nova Zelandija pa dejansko spada med bolj razvite države.
\newline
Lahko torej rečemo, da število športnikov na prebivalca v grobem lahko povežemo z gospodarsko razvitostjo držav, saj so tem športnikom omogočeni boljši pogoji za treniranje in tako lahko lažje dosežejo zahtevane norme. Obstaja pa seveda nekaj zgoraj navedenih izjem.
}
\end{enumerate}
\includepdf[pages={1}]{../slike/skupine.pdf}

% % \includegraphics{../slike/naselja.pdf}
\section{Zaključek}
Ob koncu projekta lahko pridemo do zaključka, da Olimpijske igre postajajo vedno večji dogodek, saj se z leti povečuje število športov, disciplin in dogodkov, s tem pa tudi število sodelujočih držav in športnikov. K temu veliko pripomore predvsem boljša tehnologija, ki omogoča mobilnost in izboljšuje pogoje za treniranje in pripravljanje na OI. Tudi v prihodnosti lahko pričakujemo naraščanje števil in spremembo na zemljevidu, ki prikazuje število šprtnikov na populacijo. Olimpijske igre bodo torej tudi v prihodnosti dogodek, ki povezuje svet.

\newpage
\begin{thebibliography}{9}
\bibitem{1}
  \url{http://www.naturalearthdata.com/features/}\\
  {Accessed: 23-02-2015}
\bibitem{2}
  \url{http://en.wikipedia.org/wiki/List_of_Olympic_Games_host_cities}\\
  {Accessed: 23-02-2015}
\bibitem{3}
  \url{http://en.wikipedia.org/wiki/1896_Summer_Olympics}\\
  {Accessed: 23-02-2015}
\bibitem{3}
  \url{http://en.wikipedia.org/wiki/1900_Summer_Olympics}\\
  {Accessed: 23-02-2015}
\bibitem{3}
  \url{http://en.wikipedia.org/wiki/1904_Summer_Olympics}\\
  {Accessed: 23-02-2015}
\bibitem{3}
  \url{http://en.wikipedia.org/wiki/1908_Summer_Olympics}\\
  {Accessed: 23-02-2015}
\bibitem{3}
  \url{http://en.wikipedia.org/wiki/1912_Summer_Olympics}\\
  {Accessed: 23-02-2015}
\bibitem{3}
  \url{http://en.wikipedia.org/wiki/1920_Summer_Olympics}\\
  {Accessed: 23-02-2015}
\bibitem{3}
  \url{http://en.wikipedia.org/wiki/1924_Summer_Olympics}\\
  {Accessed: 23-02-2015}
\bibitem{3}
  \url{http://en.wikipedia.org/wiki/1928_Summer_Olympics}\\
  {Accessed: 23-02-2015}
\bibitem{3}
  \url{http://en.wikipedia.org/wiki/1932_Summer_Olympics}\\
  {Accessed: 23-02-2015}
\bibitem{3}
  \url{http://en.wikipedia.org/wiki/1936_Summer_Olympics}\\
  {Accessed: 23-02-2015}
\bibitem{3}
  \url{http://en.wikipedia.org/wiki/1948_Summer_Olympics}\\
  {Accessed: 23-02-2015}
\bibitem{3}
  \url{http://en.wikipedia.org/wiki/1952_Summer_Olympics}\\
  {Accessed: 23-02-2015}
\bibitem{3}
  \url{http://en.wikipedia.org/wiki/1956_Summer_Olympics}\\
  {Accessed: 23-02-2015}
\bibitem{3}
  \url{http://en.wikipedia.org/wiki/1960_Summer_Olympics}\\
  {Accessed: 23-02-2015}
\bibitem{3}
  \url{http://en.wikipedia.org/wiki/1964_Summer_Olympics}\\
  {Accessed: 23-02-2015}
\bibitem{3}
  \url{http://en.wikipedia.org/wiki/1968_Summer_Olympics}\\
  {Accessed: 23-02-2015}
\bibitem{3}
  \url{http://en.wikipedia.org/wiki/1972_Summer_Olympics}\\
  {Accessed: 23-02-2015}
\bibitem{3}
  \url{http://en.wikipedia.org/wiki/1976_Summer_Olympics}\\
  {Accessed: 23-02-2015}
\bibitem{3}
  \url{http://en.wikipedia.org/wiki/1980_Summer_Olympics}\\
  {Accessed: 23-02-2015}
\bibitem{3}
  \url{http://en.wikipedia.org/wiki/1984_Summer_Olympics}\\
  {Accessed: 23-02-2015}
\bibitem{3}
  \url{http://en.wikipedia.org/wiki/1988_Summer_Olympics}\\
  {Accessed: 23-02-2015}
\bibitem{3}
  \url{http://en.wikipedia.org/wiki/1992_Summer_Olympics}\\
  {Accessed: 23-02-2015}
\bibitem{3}
  \url{http://en.wikipedia.org/wiki/1996_Summer_Olympics}\\
  {Accessed: 23-02-2015}
\bibitem{3}
  \url{http://en.wikipedia.org/wiki/2000_Summer_Olympics}\\
  {Accessed: 23-02-2015}
\bibitem{3}
  \url{http://en.wikipedia.org/wiki/2004_Summer_Olympics}\\
  {Accessed: 23-02-2015}
\bibitem{3}
  \url{http://en.wikipedia.org/wiki/2008_Summer_Olympics}\\
  {Accessed: 23-02-2015}
\bibitem{3}
  \url{http://en.wikipedia.org/wiki/2012_Summer_Olympics}\\
  {Accessed: 23-02-2015}
\bibitem{3}
  \url{http://en.wikipedia.org/wiki/2016_Summer_Olympics}\\
  {Accessed: 23-02-2015}
\bibitem{3}
  \url{http://en.wikipedia.org/wiki/1924_Winter_Olympics}\\
  {Accessed: 23-02-2015}
\bibitem{3}
  \url{http://en.wikipedia.org/wiki/1928_Winter_Olympics}\\
  {Accessed: 23-02-2015}
\bibitem{3}
  \url{http://en.wikipedia.org/wiki/1932_Winter_Olympics}\\
  {Accessed: 23-02-2015}
\bibitem{3}
  \url{http://en.wikipedia.org/wiki/1936_Winter_Olympics}\\
  {Accessed: 23-02-2015}
\bibitem{3}
  \url{http://en.wikipedia.org/wiki/1948_Winter_Olympics}\\
  {Accessed: 23-02-2015}
\bibitem{3}
  \url{http://en.wikipedia.org/wiki/1952_Winter_Olympics}\\
  {Accessed: 23-02-2015}
\bibitem{3}
  \url{http://en.wikipedia.org/wiki/1956_Winter_Olympics}\\
  {Accessed: 23-02-2015}
\bibitem{3}
  \url{http://en.wikipedia.org/wiki/1960_Winter_Olympics}\\
  {Accessed: 23-02-2015}
\bibitem{3}
  \url{http://en.wikipedia.org/wiki/1964_Winter_Olympics}\\
  {Accessed: 23-02-2015}
\bibitem{3}
  \url{http://en.wikipedia.org/wiki/1968_Winter_Olympics}\\
  {Accessed: 23-02-2015}
\bibitem{3}
  \url{http://en.wikipedia.org/wiki/1972_Winter_Olympics}\\
  {Accessed: 23-02-2015}
\bibitem{3}
  \url{http://en.wikipedia.org/wiki/1976_Winter_Olympics}\\
  {Accessed: 23-02-2015}
\bibitem{3}
  \url{http://en.wikipedia.org/wiki/1980_Winter_Olympics}\\
  {Accessed: 23-02-2015}
\bibitem{3}
  \url{http://en.wikipedia.org/wiki/1984_Winter_Olympics}\\
  {Accessed: 23-02-2015}
\bibitem{3}
  \url{http://en.wikipedia.org/wiki/1988_Winter_Olympics}\\
  {Accessed: 23-02-2015}
\bibitem{3}
  \url{http://en.wikipedia.org/wiki/1992_Winter_Olympics}\\
  {Accessed: 23-02-2015}
\bibitem{3}
  \url{http://en.wikipedia.org/wiki/1994_Winter_Olympics}\\
  {Accessed: 23-02-2015}
\bibitem{3}
  \url{http://en.wikipedia.org/wiki/1998_Winter_Olympics}\\
  {Accessed: 23-02-2015}
\bibitem{3}
  \url{http://en.wikipedia.org/wiki/2002_Winter_Olympics}\\
  {Accessed: 23-02-2015}
\bibitem{3}
  \url{http://en.wikipedia.org/wiki/2006_Winter_Olympics}\\
  {Accessed: 23-02-2015}
\bibitem{3}
  \url{http://en.wikipedia.org/wiki/2010_Winter_Olympics}\\
  {Accessed: 23-02-2015}
\bibitem{3}
  \url{http://en.wikipedia.org/wiki/2014_Winter_Olympics}\\
  {Accessed: 23-02-2015}
\bibitem{3}
  \url{http://en.wikipedia.org/wiki/2018_Winter_Olympics}\\
  {Accessed: 23-02-2015}
\end{thebibliography}

\end{document}
